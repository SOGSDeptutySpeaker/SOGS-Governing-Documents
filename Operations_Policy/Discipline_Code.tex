\section{Discipline Manual}

\begin{longenum}[ label*=\arabic*., align=left]
\item  \textbf{Introduction and Purpose}
\begin{longenum}[ label*=\arabic*., align=left]     
\item  The  purpose  of  the  Society's  Disciplinary  Manual  is  to  define  the  general  standard  of  conduct expected of members of the society of Graduate Students (henceforth known as the Society), provide examples  of  behaviour  that  constitutes  a  breach  of  this  standard  of  conduct  and  set  out  the disciplinary procedures that the Society will follow.
\index{Discipline!Purpose}
     \item The Society is a community of graduate students, involved in learning, teaching, research, and other activities.  The Society provides  an  environment  of  free  and  creative  inquiry  within  which  critical thinking,  humane  values,  and  practical  skills  are  cultivated  and  sustained.  It  is  committed  to  a mission and to principles that will foster excellence and create an environment where its students and staff can grow and flourish.
     \item  As  members,  students  assume  the  rights  and  responsibilities  associated  with  membership  in the  Society's  academic  and  social  community.  The  privileges  granted  to  each  member  are conditional  upon  the  fulfillment  of  this  responsibility  and  members  must  familiarize  themselves with the Society regulations and the conduct expected of them while studying at the University.
     \item Members are reminded that they are equally responsible for observing the standard of conduct 
set out in this Code when using any electronic communication devices to send or post messages or material.
     \item The Society encourages informal resolution of minor incidents.
     \item   Nothing  in  this  Code  shall  be  construed  to  prohibit  peaceful  assemblies  and  demonstrations, lawful picketing, or to inhibit free speech as guaranteed by law.
     \index{Discipline!Free Speech}
     \item Any student found responsible for misconduct is subject to the disciplinary sanctions of this Code,  regardless  of  the  action  or  inaction  of  civil  authorities.  Nothing  in  this  Code  precludes the Society from referring an individual matter to the University of Western Ontario or an appropriate law enforcement agency either before, during, or after disciplinary action is taken by the Society under  this  Code.  A  student  may  be  subject  to  criminal  prosecution  and/or  civil  proceedings notwithstanding, and in addition to, disciplinary action taken by the Society against the member under this Code.
     \index{Discipline!Referral to Law Enforcement}
     \end{longenum}
     \item \textbf{Definitions}
     \index{Discipline!Definitions}
     
     In this code:
     \begin{longenum}[ label*=\arabic*., align=left]
     \item ``Member'' is an individual that fulfills any one of the requirements of Section 5 of the Society's 
Constitution
     \item ``Ombudsperson'': shall be interpreted as the Society's Ombudsperson
     \item ``Appeals  Review  Commission''  and  ``ARC'':  Shall  be  interpreted  as  the  Commission  that  is defined under Society Bylaw 9.6.2.
     \item ``Premises  of  the  University  or  its  Affiliated  University  Colleges''  includes  lands,  buildings  and grounds of the University and its Affiliated University Colleges and other places or facilities used for the provision of the University’s courses, programs or services
     \item ``Society sponsored program, event or activity'' is a program, event or activity that  is hosted, sponsored,  or  organized  by  the Society  and  includes,  but  is  not  limited  to  organized  trips,  the  Grad-Club, Western Research Forum.

     \end{longenum}
\item \textbf{Relationship to Other University Policies and Codes}
     \begin{longenum}[ label*=\arabic*., align=left]
           \item If a member's conduct could be considered a breach of this Code and also a breach of its Conflict  of  Interest  Bylaw  and  Policy, Society, at its discretion, may proceed under the Code or under the aforesaid Bylaws and Policies. A student may not be penalized under both the Code and these Policies for the same conduct
           \index{Discipline!Jeopardy}
     \end{longenum}
\item \textbf{Jurisdiction}
     \begin{longenum}[ label*=\arabic*., align=left]
           \item This Code applies to:
                      \index{Discipline!Area of Application}

                \begin{longenum}[ label*=\arabic*., align=left]
                \item conduct  that  occurs  on  the  premises  of  the  University  or  its  Affiliated  University Colleges; 
                \item conduct that occurs at a Society sponsored or sanctioned program, event, or activity, whether the program, event, or activity is on campus or off-campus; and
                \item other off-campus conduct,
                  \begin{longenum}[ label*=\arabic*., align=left]
                       \item when the individual is acting as a designated representative of the Society or when the individual holds out that they are a representative of the Society 
                       \item that has, or might reasonably be seen to have an adverse effect on, interfere with, or threaten the proper functioning of the Society, its mission, the rights of a member of the Society community to use and enjoy the University’s learning and working  environments,  or  that  raises  concerns  for  the  safety  or  security  of  an individual   or   individuals   while   on   campus   or  while   participating   in   Society programs, events or activities.
                  \end{longenum}
                \end{longenum}
\item Graduate  students  are  subject  to  the  provisions  of  this  Code  except  when  acting  in  their capacity as Graduate Teaching Assistants.
     \end{longenum}
\item \textbf{The following list sets out specific examples of prohibited conduct}. This list is illustrative only and is not intended to define misconduct in exhaustive or exclusive terms

 \begin{longenum}[ label*=\arabic*., align=left]
                       \item Disruption:
                       
                        \index{Discipline!Prohibited Conduct!Disruption}
                       By  action,  threat,  written  material,  or  by  any  means  whatsoever,  disrupting  or  obstructing any Society activities, including a Society sponsored or sanctioned program, event or activity, or other authorized activities on premises of the University or its Affiliated University Colleges, or the right of another person to carry on their legitimate activities, or to speak or to associate with others. Society  activities  include,  but  are  not  limited  to,  research,  studying,  sports  and  recreation, administration and meetings
                
                  
                  \item Misconduct Against Persons and Dangerous Activity
                 \begin{longenum}[ label*=\arabic*., align=left]
                  \index{Discipline!Prohibited Conduct!Misconduct against Persons}
                       \item Any assault, harassment, intimidation, threats or coercion.
                       \item Conduct that threatens or endangers the health or safety of any person.
                       \item Contravention  of  The  University  of  Western  Ontario  Non-Discrimination/Harassment Policy.
                       \item Knowingly  (which  includes  when  one  should  reason ably  have  known)  creating  a condition that endangers the health, safety, or well-being of any person.
                       \item Engaging in conduct that is, or is reasonably seen to be, humiliating or demeaning to another  person  or  coercing,  enticing  or  inciting  a  person  to commit  an  act  that  is,  or  is reasonably seen to be, humiliating or demeaning to that person or to others
\end{longenum}
             \item Misconduct Involving Property
                 \begin{longenum}[ label*=\arabic*., align=left]
                  \index{Discipline!Prohibited Conduct!Misconduct involving Property}

                        \item Unauthorized entry and/or presence on any premises of the Society or any premises used for Society sponsored or sanctioned programs, events or activities.
                        \item Misappropriation,  damage,  unauthorized  possession,  defacement  and/or  destruction of premises or property of the Society, or the property of others.
                        \item Use  of  Society  facilities,  equipment  or  services  contrary  to  express  instruction  or without proper authority.
                  \end{longenum}
             
               \item False Information
                  \begin{longenum}[ label*=\arabic*., align=left]
                  \index{Discipline!Prohibited Conduct!False Information}

                        \item Furnishing false information.
                        \item Forging,   altering   or   misusing   any   document,   record,   card   or   instrument   of identification.
                  \end{longenum}

                  \item Contravention of Society Regulations

                  \begin{longenum}[ label*=\arabic*., align=left]
                  \index{Discipline!Prohibited Conduct!Contravention of Society Regulations}

                        \item  Violation of written Society policies, rules or regulations.
                  \end{longenum}
                  \item Contravention of Other Laws

                  \begin{longenum}[ label*=\arabic*., align=left]
                  \index{Discipline!Prohibited Conduct!Contravention of Other Laws}

                        \item Contravention of any provision of the Criminal Code or any other federal or provincial statute or municipal by-law.
                  \end{longenum}
                  \item Other

                  \begin{longenum}[ label*=\arabic*., align=left]
                  \index{Discipline!Prohibited Conduct!Incitement}
                  \index{Discipline!Prohibited Conduct!Refusal to Comply with Sanctions}

                        \item Aiding or encouraging others in the commission of an act prohibited under this Code or attempting to commit an act prohibited under this Code.
                        \item Failure  to  comply  with  any  sanction  imposed  by  the Society  for  misconduct  under  this Code
                  \end{longenum}
  \end{longenum}
  
  \item \textbf{Sanctions}
  
  The Society  may  impose  one  or  more  sanctions  for  misconduct,  as  per  Bylaw  19.  The  sanctions imposed should be proportional to the type of misconduct. The most serious types of misconduct will  merit  the  most  serious  sanctions.  In  considering  an  appropriate  sanction,  the Society's  primary focus must be to ensure the safety and security of the Society, its members, and visitors
  
  
  
  
  
\item \textbf{Interim Measures}

  \begin{longenum}[ label*=\arabic*., align=left]
  \index{Discipline!Temporary Exclusions}

      \item Temporary Exclusions
                        
              An Official Liaison may exclude a member from a committee that they oversee if they believe on 
reasonable  grounds  that  the  student’s  continued  presence  is  detrimental  to  good  order  or  will constitute a threat to the safety of others. Such initial exclusion shall last for the duration of the meeting and shall be reported immediately to the Ombudsperson.
   \item Interim Prohibition
    \index{Discipline!Interim Prohibition}
 The Ombudsperson may impose an interim prohibition pending an investigation and disposition of a complaint of misconduct. Interim prohibition may be imposed only:
 
  \begin{itemize}
     \item  if needed to ensure the safety and well-being of members of the Society community or preservation of Society property; 
      \item if needed to ensure the Member's own physical or emotional safety and well-being; or 
      \item if there is a reasonable apprehension that the Member poses a threat of disruption or of interference with the normal operations of the Society. 
      \end{itemize}                     
                          
      As per Bylaw 9.3, during a period of interim prohibition, a student may be denied access to specified Society facilities (including the Grad Club) and/or any other  Society  sponsored and/or  Society  sanctioned  activities  or  privileges  for which the Member might otherwise be eligible, as the Ombudsperson may determine to be appropriate. Within two working  days  following  the  imposition  of  interim  prohibition,  the  student  shall  be  informed  in writing of the reasons for the prohibition. The student shall be afforded the opportunity to respond to the allegations being made against them. If the student responds, the Ombudsperson will reassess the prohibition and either revoke or continue the prohibition pending formal disposition 
of the matter  
\end{longenum}                      
\item \textbf{Complaint Procedures}

     \begin{longenum}[ label*=\arabic*., align=left]

           \item Any  member(s)  may  submit  a  complaint  of  misconduct  against  a  member(s).  A  complaint 
should be submitted to the Ombudsperson
           \item The Ombudsperson shall not make a finding of misconduct nor impose a sanction or sanctions  against  a  student  unless  the  student  has  been  informed,  in  writing,  of  the nature  of  the  complaint,  the  facts  alleged  against  them,  and  has  been  given  a reasonable  opportunity  to  respond  to  them  and  to  submit  relevant  information.  The student  shall also  be  given  a  reasonable  opportunity  to  meet  personally  with  the Ombudsperson to discuss the matter. It is the responsibility of both parties to provide all materials    and    information    that    will    support    their    positions.  Furthermore,    the Ombudsperson  will make  reasonable  attempts  to  ascertain  the  truth  to  the  best  of  their ability. 
           
           \begin{longenum}[ label*=\arabic*., align=left]
           \item If the Ombudsperson feels that they are not in a position to fairly rule on a complaint  due  to  conflict  of  interest  or  recuse themselves for any reason, they can forward  the  complaint  to  a  member  of  the  ARC,  chosen  at  random,  by  the Speaker, to act as the Ombudsperson for this matter only. 
           \end{longenum}
           \item If the Ombudsperson concludes that there has been misconduct, the Ombudsperson may impose an appropriate sanction or sanctions
           \item If  the  student  does  not  respond  to  the  allegation  or  does  not  meet  with  the Ombudsperson   after   having   been   given   a   reasonable   opportunity   to   do   so,   the Ombudsperson  may  proceed  to  dispose  of  the  complaint  without  such  a  response  or meeting.
           \item At  all  meetings  with  the  Ombudsperson,  both  parties  may  be  accompanied  by  a colleague  of  their  choosing.  Legal  representation  is  not  permitted  at  this  stage;  it  is permitted at the appeal stage.
          \item In determining an appropriate sanction or sanctions, the Ombudsperson may take into account  any  previous  findings  of  misconduct.  The  Ombudsperson  may  direct  that  a sanction be held in abeyance if a member's registration at the University is interrupted for any reason.
           \item The decision of the Ombudsperson, with reasons, shall be communicated in writing to the member. If there is a finding of misconduct, a copy of the decision will be retained in the  Society  Office.  A  copy  of  the  decision  shall  be  provided  on  a  need-to-know  basis  to administrative  units  (e.g.  Executive  Officers  and  Non-Executive  Officers).  The  Speaker and  the  relevant  Executive  is  responsible  for  the  implementation  of  any  decision  made under the Code.
          \item All notices and other communications from the Ombudsperson to the student or any other member of the University community, shall be by personal delivery, campus mail, e-mail, priority post, courier, or registered mail. 
          \item Complaints  of  misconduct  shall  be  reported,  investigated,  and  decided  in  a  timely manner.
          \item The  Ombudsperson  shall  report  annually  to  the  Annual General Meeting,  summarizing  the  number  of complaints received, number of complaints investigated, and the general nature of the of matters investigated.
          \item After five (5) years the files will be expunged.
   
       \end{longenum}
\item \textbf{Appeals}             
     \begin{longenum}[ label*=\arabic*., align=left]
		\item A  student  may  appeal  an  Ombudsperson's  finding  of  misconduct  to  the  Appeals  Review Commission on one or more of the following grounds:
        \index{Discipline!Appeals!Grounds for Appeal}
		\begin{longenum}[ label*=\arabic*., align=left]
			\item that  there  was  a  serious  procedural  error  in  the  hearing  of  the  complaint  by  the Ombudsperson which was prejudicial to the appellant;
			\item that  new  evidence,  not  available  at  the  time  of  the  earlier  decision,  has  been discovered, which casts doubt on the correctness of the decision;
			\item that  the  Ombudsperson  did  not  have  the  authority  under  this  Code  to  reach  the decision or impose the sanctions they did.
            
      \end{longenum}

	\item Filing an Appeal Application will not stay the implementation of any sanctions imposed
           \index{Discipline!Appeals Review Commission!Granting or Refusing Appeal} 
           \index{Appeals Review Commission!Granting or Refusing Appeal} 
    \item The ARC may:
		\begin{longenum}[ label*=\arabic*., align=left]
        \item Deny the appeal.
		\item In  the  case  of  an  appeal  under  section  1(a)  or  (b),  grant  the  appeal  and  direct  the Ombudsperson to rehear the matter or reconsider some pertinent aspect of its decision, and may include recommendations relating to the conduct of any rehearing, or quash the original decision.
		\item In the case of an appeal under section 1(c), grant the appeal and quash the original decision.
      \end{longenum}
	\item The  right  to  be  represented  by  legal  counsel  will  be  accorded  to  the  principal  parties  to  the appeal at this level. ARC also reserves the right to retain counsel only when the appellant is being represented by legal counsel with the approval of the executive.
    \begin{longenum}[ label*=\arabic*., align=left]
			\item If the appellant wishes to bring legal counsel to the appeal, the appellant must inform 
the ARC in writing 72 hours prior to the meeting
\index{Discipline!Appeals!Legal Council}

      \end{longenum}
      \item The parties must bear all their own legal expenses, if any. ARC will not order the Society to pay all or part of the appellant's costs nor will it order the appellant to pay all or part of the societys' costs
      \item Composition: 
            \index{Discipline!Appeals REview Commission!Composition}
            \index{Appeals Review Commission!Composition}
                  
Membership  for  the  ARC  panel  will  be  drawn  from  the  Appeals  Review  Commission,  with  its membership drawn from Bylaw 9.2
      \item Procedures
      \index{Discipline!Appeals!Procedure}
      
An  Appeal Application  must  be  filed  with  the  Society  Office  Manager  in  a  sealed  envelope addressed  to  the  ARC  Chair  within  two  weeks  after  a  decision  has  been  issued  by  the Ombudsperson. The Application must contain a copy of the decision, the grounds for the appeal, the  outcome  sought,  a  full  statement  supporting  the  grounds  for  the  appeal,  the  name  of  legal counsel  or  agent,  if  any,  and  any  relevant  documentation  in  support  of  the  appeal.  Where  the basis of the appeal is new evidence, such new evidence shall be described comprehensively and the names of any witnesses shall be provided.
    \begin{longenum}[ label*=\arabic*., align=left]
			\item The Appeals Review Commission Chair must inform the Commission that an Appeal has been received
            \index{Appeals Review Commission!Chairperson!Receiving Appeal}
      \end{longenum}
\item An Appeal Application will not be accepted by the Appeals Review Commission Chair if incomplete or not filed within the time period specified in section 8 above. Exceptions to the time limit for filing an appeal are at the discretion of the Appeals Review Commission upon written application of the student
  \index{Discipline!Appeals!Incomplete Appeal}
   \index{Appeals Review Commission!Incomplete Appeal}
\item Parties  to  an  appeal  are  the  student  against  whom  the  decision  has  been  made  (Appellant) and the Ombudsperson (Respondent)   
  \index{Discipline!Appeals!Parties to the Appeal}

\item The Respondent shall file a concise written reply to the Appeal Application with the Speaker within five (5) business days of receiving the documents. A copy of the reply shall be provided to the Appellant

\item Upon receipt of  an Appeal Application, the Appeals Review Commission Chair shall:
   \index{Appeals Review Commission!Duties of the Chairperson}
      \index{Chairperson!Appeals Review Commission!Duties}
 \begin{longenum}[ label*=\arabic*., align=left]
			\item Constitute a Panel of at least three (3) members including the chairperson. If the chairperson recuses themself from the proceedings, the panel must elect an interim chairperson from within their ranks.
               \index{Appeals Review Commission!Creating a Panel}
               \index{Appeals Review Commission!Recusal}
			   \index{Quorum!Appeals Review Commission!Panel}
            
            \begin{longenum}[ label*=\arabic*., align=left]
			\item  Member(s) from the same department as either of the parties shall recluse themselves from the panel.
      \end{longenum}
     \item facilitate the scheduling of the initial meeting of the Panel.       
     \end{longenum}
 	\item Subject  to  the  requirements  set  out  herein,  the  Panel  shall  determine  its  own policy document,  subject  to  approval  by  Council,  as  the  Commission  deems  necessary  and  proper  to ensure a fair and expeditious proceeding. The Panel is bound by neither strict legal procedures nor  strict  rules  of  evidence.  It shall  proceed  fairly  in  its  disposition  of  the  appeal,  ensuring  that both  parties  are  aware  of  the evidence  to  be  considered,  are  given  copies  of  all  documents considered by the Panel, and are given an opportunity to be heard during the process.   
    \index{Policy Document!Appeals Review Commission}
    \item The  Panel  may  summarily  dismiss  an  appeal  if  the  Appeal  Application  does  not,  in  the judgment  of  the  Panel,  raise  a  valid  ground  of  appeal  or  does  not  assert  evidence  capable  of supporting a valid ground.
	\item The Panel shall hold an oral hearing if any party and/or the ARC Panel requests one
    \item Both parties may petition the ARC Panel to make the oral hearing open to the society and in camera.
   \index{Appeals Review Commission!Oral Hearing}

		\begin{longenum}[ label*=\arabic*., align=left]
			\item  The decision to make an oral hearing open to the Society rests with the ARC Panel
     	 \end{longenum}
	\item While an attempt shall be made to schedule an oral hearing at a time convenient to the Panel and  the  parties,  a  request  by  a  party  for  a  lengthy  delay  in  the  scheduling  of  the  hearing,  or  a postponement  of  a  scheduled  hearing,  will  be  granted  by  the  Chair  only  in  exceptional circumstances.  Oral  hearings  will  ordinarily  be  held  within  six  weeks  of  filing  of  the  Appeal Application.  In  the  case  of  an  oral  hearing,  if  the  ARC  Chair  is  unable  to  contact  the  Appellant within a reasonable time to schedule a hearing, the Appellant will be notified by registered mail at the  address  on  the  Appeal  Application  of  the  deadline  by  which  they must  contact  the  ARC Chair  to  arrange  a  hearing.  If  the  Appellant  has  not  contacted  the  ARC  Chair  by  the  specified deadline, the appeal will be deemed to be abandoned.
	\item Each party to an oral hearing shall be sent a Notice of Hearing, setting out the time, place, and  purpose  of  the  hearing.  If  a  party  does  not  attend,  the  Panel  may  proceed  in  the  party's absence.
    \item Each  member  of  a  Panel,  including  the  Chair,  shall  vote.  There  shall be  no  abstentions.  A majority of positive votes is required to grant an appeal.
       \index{Appeals Review Commission!Voting Procedure}
	\item The  decision,  with  reasons,  shall  be  filed  with  the  Speaker  and  copies  shall  be  sent  to  the parties  to  the  proceedings  as  well  as  to  others  with  a  legitimate  need  to  know  (e.g.  Relevant Executive) 
      \end{longenum}
\item \textbf{Review of Code}

The Policy Committee shall review the Code within twelve to twenty-four of initial implementation and when deemed necessary thereafter.
\index{Discipline!Review of Discipline Manual}
\index{Policy Committee!Review of Discipline Manual}
\end{longenum}